\documentclass[12pt]{report}
\usepackage{amsmath}
\usepackage{ragged2e}
\usepackage{graphicx}
\usepackage{multirow}
\usepackage{amsfonts}
\usepackage{mathtools}
\usepackage[absolute,overlay]{textpos}
\graphicspath{ {./img/} }
\setlength\parindent{0pt}

\makeatletter
\newcommand*{\Xbar}{}%
\DeclareRobustCommand*{\Xbar}{%
	\mathpalette\@Xbar{}%
}
\newcommand*{\@Xbar}[2]{%
	% #1: math style
	% #2: unused (empty)
	\sbox0{$#1\mathrm{X}\m@th$}%
	\sbox2{$#1X\m@th$}%
	\rlap{%
		\hbox to\wd2{%
			\hfill
			$\overline{%
				\vrule width 0pt height\ht0 %
				\kern\wd0 %
			}$%
		}%
	}%
	\copy2 %
}
\makeatother

\newcommand{\thn}{\textsuperscript{th}}

\newcommand{\twopartdef}[4]
{
	\left\{
	\begin{array}{ll}
		#1 & \mbox{} #2 \\
		#3 & \mbox{} #4
	\end{array}
	\right.
}

\newcommand{\threepartdef}[6]
{
	\left\{
	\begin{array}{lll}
		#1 & \mbox{} #2 \\
		#3 & \mbox{} #4 \\
		#5 & \mbox{} #6
	\end{array}
	\right.
}

\begin{document}

\Large
\centering
AMS310 Homework 4

\justify
\normalsize

Kuba Gasiorowski\\
ID: 109776237\\

\noindent \textbf{1a.} From the CLT, if the sample size is large, then the distribution of the sample mean is approximated by the normal distribution. This distribution has the following $\sigma_{\Xbar}$ and $\mu_{\Xbar}$:
\begin{align*}
\mu_{\Xbar} = \mu &= \boxed{31}\\
\sigma_{\Xbar} = \frac{\sigma}{\sqrt{n}} &= \boxed{0.5}
\end{align*}

\noindent \textbf{b.} Find $P(\Xbar > 31.5)$ given $\mu = 31,\;\sigma = 5 \text{ and } n = 100$.


\begin{align*}
	P(\Xbar > 31.5) &= 1 - P(\Xbar < 31.5)&&\\
	&= 1 - P\left(Z < \frac{31.5 - \mu}{\frac{\sigma}{\sqrt{n}}}\right)\\
	&= 1 - P\left(Z < \frac{31.5 - 31}{\frac{5}{\sqrt{100}}}\right)\\
	&= 1 - P(Z < 1)\\
	&= 1 - \Phi(1)\\
	&= 1 - .84134\\
	&= \boxed{0.15866}
\end{align*}

\pagebreak
\noindent \textbf{2.} Given: $\mu=69.5, \; \sigma = 3$; find:

\noindent \textbf{a.} $P(68 < \Xbar < 70) \text{ with } n = 20$\\

\begin{align*}
	P(68 < \Xbar < 70) &= P\left(\frac{68 - \mu}{\frac{\sigma}{\sqrt{n}}} < Z < \frac{70 - \mu}{\frac{\sigma}{\sqrt{n}}}\right)\\
	&= P\left(\frac{68 - 69.5}{\frac{3}{\sqrt{20}}} < Z < \frac{70 - 69.5}{\frac{3}{\sqrt{20}}}\right)\\
	&= P(-2.24 < Z < 0.75)\\
	&= \Phi(0.75) - \Phi(-2.24)\\
	&= .7734 - .0125 \\
	&= \boxed{.7609}
\end{align*}

\noindent \textbf{b.} Given $P(68.52 < \Xbar < 70.48) = .95$, find $n$.

\begin{align*}
	.95 &= P(68.52 < \Xbar < 70.48)\\
	.95 &= \Phi\left(\frac{70.48-\mu}{\frac{\sigma}{\sqrt{n}}}\right) - \Phi\left(\frac{68.52-\mu}{\frac{\sigma}{\sqrt{n}}}\right)\\
	.95 &= \Phi\left(\frac{70.48 - 69.5}{\frac{3}{\sqrt{n}}}\right) - \Phi\left(\frac{68.52-69.5}{\frac{3}{\sqrt{n}}}\right)\\
\end{align*}

\noindent Now that we have everything in terms of $n$, we solve. But since this calculation is tedious, I just plugged the formula into my calculator and guess-and-checked until I arrived at \boxed{n = 36.} Plugging $n$ back in, I verified that this is in fact correct.\\

\pagebreak
\noindent \textbf{c.} First we find $P(X < 68)$:

\begin{align*}
	P(X < 68) &= P\left(Z < \frac{68 - 69.5}{3}\right)\\
	&= \Phi(-0.5)\\
	&= .30854
\end{align*}

\noindent Now we find $P(Y \geq 25)$:

\begin{align*}
	P(Y \geq 25) &= 1 - P(Y = 24)\\
	&= 1 - \left[{100 \choose 24} \cdot 0.30854^{24} \cdot (1-0.30854)^{76}\right]\\
	&= \boxed{0.0294}
\end{align*}

\noindent \textbf{3a.} 

\begin{align*}
	\text{Mean of }X = E(X) &= \int_{-\infty}^{\infty}xf(x)dx\\
	&= \int_{0}^{1}x\left(\frac{3x^2}{2} + x\right)dx\\
	&= \int_{0}^{1}\frac{3x^3}{2} + x^2dx\\
	&= \left[\frac{3x^4}{8} + \frac{x^3}{3}\right]_0^1\\
	&= \boxed{\frac{17}{24} \approx 0.708}
\end{align*}

\pagebreak
\noindent \textbf{b.} 

\begin{align*}
	\text{Var of X} &= E(X^2) - E(X)^2\\
	&= \int_{-\infty}^{\infty} x^2 f(x)dx - \mu^2\\
	&= \int_{0}^{1} x^2\left(\frac{3x^2}{2} + x\right)dx - 0.708^2\\
	&= \int_{0}^{1}\frac{3x^4}{2} + x^3dx - 0.708^2\\
	&= 0.55 - 0.708^2\\
	&= \boxed{0.0487}
\end{align*}

\noindent \textbf{c.} 

\begin{align*}
	P(0.7 < \Xbar < 0.75) &= P\left(\frac{0.7 - \mu}{\frac{\sqrt{\sigma^2}}{\sqrt{n}}} < Z < \frac{0.75 - \mu}{\frac{\sqrt{\sigma^2}}{\sqrt{n}}}\right)\\
	&= P\left(\frac{0.7 - 0.708}{\frac{\sqrt{0.0487}}{\sqrt{100}}} < Z < \frac{0.75 - 0.708}{\frac{\sqrt{0.0487}}{\sqrt{100}}}\right)\\
	&= P(-0.36 < Z < 1.90)\\
	&= \Phi(1.90) - \Phi(-0.36)\\
	&= 0.9713 - 0.3594\\
	&= \boxed{0.6119}
\end{align*}

\pagebreak
\noindent \textbf{4a.} From the $\chi^2$ table, for values of $df = 14$ and $\alpha = 0.05$, we get a value of 23.685. Then:

\begin{align*}
	\frac{(n-1)s^2}{\sigma^2} &= \chi^2_{\alpha,\; n-1}\\
	s^2 &= \frac{\chi^2_{\alpha,\;n-1} \cdot \sigma^2}{n-1}\\
	&= \frac{(23.68) \cdot 3}{14}\\
	&= \boxed{5.0753}
\end{align*}

\noindent \textbf{b.} 
\begin{verbatim}
> 3*qchisq(.95, 14)/14
[1] 5.0753
\end{verbatim}

\noindent \textbf{5a.} 
\begin{verbatim}
> pf(1.7298, 7, 11)
[1] 0.799998
\end{verbatim}

\noindent \textbf{b.}
\begin{verbatim}
> pf(1.7298, 7, 11) - pf(0.3726, 7, 11)
[1] 0.6999914
\end{verbatim}

\noindent \textbf{c.}
\begin{verbatim}
> A <- c(1,2,3,4)
> 1 - pf(A, 7, 11)
[1] 0.47950897 0.14625028 0.05060951 0.02029701
\end{verbatim}

\end{document}


